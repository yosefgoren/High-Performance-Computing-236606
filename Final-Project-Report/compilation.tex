\subsection*{Objective}
The objective of this part is to make the most out of
the standard compiler technologies and achive a good serial CPU performance.
To achive this task, different compilers with different optimization options
were considered and trailed.

\subsection*{Compilers}
Mutiple Fortran compilers have been trailed throught
the development. Namely: 'GNU gfortran (7.4.0)', 'Intel ifort (2021.8.0)' and 'Intel ifx (2023.0.0)'.\\
In the table, the average runtime of a single simulation cycle is shown as a function of the compiler
and the optimization level - The test is done on the original code.
The calculation excludes the first cycle, since it often has
a longer runtime - likely due to 'cold' caches.\\
This runtime for only one cycle should be negligable
when many cycles are run.
\begin{center}
\begin{tabular}{| c |  c  c  c |}
    \hline
                 & ifort    & ifx       & gfortran\\
    \hline
    O0 (none)    & 15.02    & 11.76     & 12.4 \\ 
    O2 (default) & 5.08     & 4.93     & 12.41 \\  
    Ofast        & 4.73     & 4.92      & 12.41    \\
    \hline
\end{tabular}
\end{center}


Running all three compilers shows that 'Intel ifort' is the fastest,
while being marginally better than 'Intel ifx' and significantly better than 'GNU gfortran'.\\
For the rest of the project, 'Intel ifort' will be used.

\subsection*{Flag Optimizations}
After selecting the 'ifort' compiler with the 'Ofast' optimization level,
a few specific flags were trailed. So before any of these took effect
the runtime was 4.73 seconds.\\
\begin{itemize}
    \item The '-ipo' flag:\\
    This flag enables interprocedural optimization between
    procedures defined in different files. This might be useful here since the main simulation
    loop takes place in many different sources.\\
    The result was a runtime of 3.39 seconds - equivalent to a $\times 1.39$ speedup.
    
    \item The '-xHost' flag:\\
    This flag requests the compiler to generate code which
    is best optimized for the current host machine.\\
    It enables the compiler to take into account the specific
    hardware which is available.
    The resulting runtime was 3.8 seconds - equivalent to a $\times 1.24$ speedup.\\
    
    \item The '-g' flag (check removed):\\
    The '-g' adds debug information and other compilation changes to make the output easier to debug, 
    removing it did not make any noticable difference (but did make a difference when an explicit
    optimization level flag is not provided. In this case it changes the default optimization level from '-O2' to '-O0').
\end{itemize}
When combining the '-ipo' and '-xHost' flags,
the we get a runtime of 2.44 seconds - equivalent to a $\times 1.94$ speedup.\\

By examining a run of this version with Vtune 'HPC' mode,
we can see that the 'calculate\_derivatives' procedure
has seen vectorization
\footnote{
    In the procedue a small conditional branch was meant to
    decrease the workload by ignoring cells with negligable volume,
    however, this branch prevented vectorization - and was removed.
}
- likely due to the '-xHost' flag.\\
